% Example for Carleton student papers
% Author: Andrew Gainer-Dewar, 2013
% This work is licensed under the Creative Commons Attribution 4.0 International License.
% To view a copy of this license, visit http://creativecommons.org/licenses/by/4.0/ or send a letter to Creative Commons, 444 Castro Street, Suite 900, Mountain View, California, 94041, USA.
\documentclass[twoside]{article}
\usepackage{ccpaper}

% The Latin Modern font is a modernized replacement for the classic
% Computer Modern. Feel free to replace this with a different font package.
\usepackage{lmodern}

% These commands tell the ccpaper package about your document.
% They are used to set up the header.
\title{Example paper}
\subtitle{A demonstration of the use of the \texttt{ccpaper} style} % \subtitle{} is optional. Simply omit it if you don't want one.
\author{Lori T.~Jeremiah \and Ray B.~Langstrom} % The \and command takes care of putting the names side-by-side.
\date{February 31, 2014}
\prof{Professor M.~Holmes}
\course{HIST-314: History of Magic}

% To enable double spacing, uncomment this line:
% \doublespacing

% Load in biblatex
% To use a different bibliography style, just change "numeric" to
% your preferred style (mla for MLA style, alphabetic for Author-Year
% style, etc.) There are a lot of options; check the BibLaTeX documentation.
\usepackage[backend=bibtex,style=numeric]{biblatex}
% Select the bibliography file
\addbibresource{sources.bib}

\begin{document}
% Calling \maketitle tells the package to print out the title using
% the information you configured above.
\maketitle{}

Test text goes here.
Suppose this were long enough to break onto a second line---wouldn't that be grand?
Well, to everyone's delight, it just so happens that it shall!
In fact, if we're very lucky, we might even get to see \emph{three} lines.

\section*{Sections and other divisions}
Not all papers will have sections, of course.
However, if yours does, you should be sure to mark them using \LaTeX{}'s commands for that purpose.
Otherwise, things may end up looking very strange.

\section*{Citation and reference}
A good academic paper should be chock-full of citations.
For example, any \LaTeX{} guide owes a great debt to the \enquote{Not so short introduction to \LaTeXe} \cite{notsoshort}.

\printbibliography
\end{document}